@c                                           -*- Texinfo -*-
@node Syscalls
@chapter System Calls

@cindex linking the C library
The C subroutine library depends on a handful of subroutine calls for
operating system services.  If you use the C library on a system that
complies with the POSIX.1 standard (also known as IEEE 1003.1), most of
these subroutines are supplied with your operating system.

If some of these subroutines are not provided with your system---in
the extreme case, if you are developing software for a ``bare board''
system, without an OS---you will at least need to provide do-nothing
stubs (or subroutines with minimal functionality) to allow your
programs to link with the subroutines in @code{libc.a}.

System routines may be provided in either reentrant or non-reentrant
versions.  The reentrant versions take an extra argument for the
reentrancy structure and have different names, formed by prepending
@samp{_} and appending @samp{_r} (e.g., @code{_close_r} instead of
@code{close}).  If newlib is configured to use reentrant system
routines, the non-reentrant versions of the system routines are provided
as cover routines.  Likewise, if newlib is configured to use
non-reentrant system routines, reentrant cover routines are provided
that achieve reentrancy by using a reserved global data block.
@xref{Reentrancy,,Reentrancy}.

@menu
* Stubs::		Definitions for OS interface
@ifset XTENSA
* Xtensa Syscalls::     System calls in the default Xtensa runtime
@end ifset
@end menu

@node Stubs
@section Definitions for OS interface
@cindex stubs

@cindex subroutines for OS interface
@cindex OS interface subroutines
This is the complete set of system definitions (primarily subroutines)
required; the examples shown implement the minimal functionality
required to allow @code{libc} to link, and fail gracefully where OS
services are not available.  

Graceful failure is permitted by returning an error code.  If newlib is
configured to use non-reentrant system routines, a minor
complication arises here: the C library must be compatible with
development environments that supply fully functional versions of these
subroutines.  Such environments usually return error codes in a global
@code{errno}.  However, the Red Hat newlib C library provides a @emph{macro}
definition for @code{errno} in the header file @file{errno.h}, as part
of its support for reentrant routines (@pxref{Reentrancy,,Reentrancy}).

@cindex @code{errno} global vs macro
The bridge between these two interpretations of @code{errno} is
straightforward: the C library routines with OS interface calls
capture the @code{errno} values returned globally, and record them in
the appropriate field of the reentrancy structure (so that you can query
them using the @code{errno} macro from @file{errno.h}).

This mechanism becomes visible when you write non-reentrant stub
routines for OS interfaces.  You must include @file{errno.h}, then
disable the macro, like this:

@example
#include <errno.h>
#undef errno
extern int errno;
@end example

@noindent
The examples in this chapter include this treatment of @code{errno}.
If you are writing reentrant stub routines, this issue is irrelevant
because the error code is stored into the reentrancy structure and 
there is no need for a global variable.

For simplicity, only the non-reentrant versions of the system routines
are shown here.  The reentrant versions are similar except for
the function names, the extra reentrancy structure argument,
and the error codes being stored into the reentrancy structure
instead of the global @code{errno} variable.

@ftable @code
@item _exit
Exit a program without cleaning up files.  If your system doesn't
provide this, it is best to avoid linking with subroutines that require
it (@code{exit}, @code{system}).  Note that there is no separate
reentrant version of this routine, even when newlib is configured to use
reentrant system routines.

@item close
Close a file.  Minimal implementation:

@example
int close(int file) @{
  return -1;
@}
@end example

@item environ
A pointer to a list of environment variables and their values.  For a
minimal environment, this empty list is adequate:

@example
char *__env[1] = @{ 0 @};
char **environ = __env;
@end example

@item execve
Transfer control to a new process.  Minimal implementation (for a system
without processes):

@example
#include <errno.h>
#undef errno
extern int errno;
int execve(char *name, char **argv, char **env) @{
  errno = ENOMEM;
  return -1;
@}
@end example

@item fcntl
File control commands.  Minimal implementation:

@example
int fcntl(int fildes, int cmd, int arg) @{
  errno = EAGAIN;
  return -1;
@}
@end example

@item fork
Create a new process.  Minimal implementation (for a system without processes):

@example
#include <errno.h>
#undef errno
extern int errno;
int fork(void) @{
  errno = EAGAIN;
  return -1;
@}
@end example

@item fstat
Status of an open file.  For consistency with other minimal
implementations in these examples, all files are regarded as character
special devices.  The @file{sys/stat.h} header file required is
distributed in the @file{include} subdirectory for this C library.

@example
#include <sys/stat.h>
int fstat(int file, struct stat *st) @{
  st->st_mode = S_IFCHR;
  return 0;
@}
@end example

@item getpid
Process-ID; this is sometimes used to generate strings unlikely to
conflict with other processes.  Minimal implementation, for a system
without processes:

@example
int getpid(void) @{
  return 1;
@}
@end example

@item gettimeofday
Get the current time.  Minimal implementation:

@example
#include <sys/time.h>
int gettimeofday(struct timeval *tv,
                 struct timezone *tz) @{
  errno = EINVAL;
  return -1;
@}
@end example

@ifclear XTENSA
@c isatty is provided by newlib for Xtensa (in libc/sys/xtensa/isatty.c).
@item isatty
Query whether output stream is a terminal.   For consistency with the
other minimal implementations, which only support output to
@code{stdout}, this minimal implementation is suggested:

@example
int isatty(int file) @{
  return 1;
@}
@end example
@end ifclear

@item kill
Send a signal.  Minimal implementation:

@example
#include <errno.h>
#undef errno
extern int errno;
int kill(int pid, int sig) @{
  errno = EINVAL;
  return -1;
@}
@end example

@item link
Establish a new name for an existing file.  Minimal implementation:

@example
#include <errno.h>
#undef errno
extern int errno;
int link(char *old, char *new) @{
  errno = EMLINK;
  return -1;
@}
@end example

@item lseek
Set position in a file.  Minimal implementation:

@example
int lseek(int file, int ptr, int dir) @{
  return 0;
@}
@end example

@item open
Open a file.  Minimal implementation:

@example
int open(const char *name, int flags, int mode) @{
  return -1;
@}
@end example

@item read
Read from a file.  Minimal implementation:

@example
int read(int file, char *ptr, int len) @{
  return 0;
@}
@end example

@item _rename
Rename a file.  For systems that do not support the @code{link} system
routine, newlib can be configured to use this optional routine for
renaming files.  Note that the reentrant version of this routine is
named @code{_rename_r}.  Minimal implementation:

@example
int _rename(char *oldpath, char *newpath) @{
  errno = EINVAL;
  return -1;
@}
@end example

@item sbrk
Increase program data space.  As @code{malloc} and related functions
depend on this, it is useful to have a working implementation.  The
following suffices for a standalone system; it exploits the symbol
@code{_end} automatically defined by the GNU linker.

@example
@group
caddr_t sbrk(int incr) @{
  extern char _end;		/* @r{Defined by the linker} */
  static char *heap_end;
  char *prev_heap_end;
 
  if (heap_end == 0) @{
    heap_end = &_end;
  @}
  prev_heap_end = heap_end;
  if (heap_end + incr > stack_ptr) @{
    write (1, "Heap and stack collision\n", 25);
    abort ();
  @}

  heap_end += incr;
  return (caddr_t) prev_heap_end;
@}
@end group
@end example

@item stat
Status of a file (by name).  Minimal implementation:

@example
int stat(char *file, struct stat *st) @{
  st->st_mode = S_IFCHR;
  return 0;
@}
@end example

@item times
Timing information for current process.  Minimal implementation:

@example
int times(struct tms *buf) @{
  return -1;
@}
@end example

@item unlink
Remove a file's directory entry.  Minimal implementation:

@example
#include <errno.h>
#undef errno
extern int errno;
int unlink(char *name) @{
  errno = ENOENT;
  return -1; 
@}
@end example

@item wait
Wait for a child process.  Minimal implementation:
@example
#include <errno.h>
#undef errno
extern int errno;
int wait(int *status) @{
  errno = ECHILD;
  return -1;
@}
@end example

@item write
Write to a file.  @file{libc} subroutines will use this
system routine for output to all files, @emph{including}
@code{stdout}---so if you need to generate any output, for example to a
serial port for debugging, you should make your minimal @code{write}
capable of doing this.  The following minimal implementation is an
incomplete example; it relies on an @code{outbyte} subroutine (not
shown; typically, you must write this in assembler from examples
provided by your hardware manufacturer) to actually perform the output.

@example
@group
int write(int file, char *ptr, int len) @{
  int todo;

  for (todo = 0; todo < len; todo++) @{
    outbyte (*ptr++);
  @}
  return len;
@}
@end group
@end example

@end ftable

@ifset XTENSA
@node Xtensa Syscalls
@section Xtensa system calls

If you are not using an operating system, the default Xtensa runtime
environment provides three options for system calls.  The first option
is the GNU low-level operating system support library (libgloss),
which provides minimal stub routines, similar to those described above
(@pxref{Stubs}).  Please refer to the @cite{Xtensa Linker Support
Packages (LSPs) Reference Manual} for instructions on modifying and
rebuilding libgloss.

The second and third options both pass some system calls to a host machine,
where they are evaluated before passing the results back to the target
system.  With this approach, your Xtensa system can access the file system
of the host machine, which is often useful for development and debugging.
The ``sim'' library uses the ``semihosting'' feature of the Xtensa
instruction set simulator (ISS) to implement system calls that pass through
to the host running the ISS.  This is the default when running on the
Xtensa ISS.  The ``gdbio'' library uses the ``File-I/O'' remote protocol
extension of the GNU debugger (GDB) to pass system calls via the On-Chip
Debugging (OCD)
interface to a host machine running GDB.  See the @cite{Xtensa Software
Development Toolkit User's Guide} for information about using the gdbio
library.  The ``File-I/O remote protocol extension'' appendix section in the
@cite{GNU Debugger User's Guide} describes the details of this feature.

Tensilica configures newlib to use reentrant system call interfaces.
The actual system call implementations may not all be reentrant; see
the @cite{Xtensa Linker Support Packages (LSPs) Reference Manual} for
details.  The system routines are listed in this section with the
reentrant function names.

@menu
* Xtensa Libgloss::		The Xtensa version of libgloss.
* Xtensa ISS Syscalls:: 	System calls supported by the ISS.
* Xtensa gdbio::		System calls supported by gdbio.
@end menu

@node Xtensa Libgloss
@subsection System calls with the libgloss library

The Xtensa libgloss library implements the system routines as described
below.  Some of the routines are not implemented; programs using
libgloss must avoid linking with functions that require those
unimplemented routines.

@table @code
@item _exit
Not implemented.  It is assumed that this function is provided
elsewhere, typically in the @samp{crt1} file.

@item _close_r
Does nothing.  Returns 0.

@item environ
Not implemented.  The default implementation in newlib is used unless
another definition is provided elsewhere.

@item _execve_r
Not implemented.

@item _fcntl_r
Not implemented.

@item _fork_r
Not implemented.

@item _fstat_r
Assumes terminal I/O and sets @code{st_mode} to @code{S_IFCHR}.
Returns 0.

@item _getpid_r
Returns 1.

@item _gettimeofday_r
Not implemented.

@item _kill_r
Calls @code{_exit} if the specified process ID is 1 (as returned by
@code{getpid}); otherwise, does nothing.  Returns 0.

@item _link_r
Not implemented.

@item _lseek_r
Sets @code{errno} to @code{ESPIPE}.  Returns -1.

@item _open_r
Sets @code{errno} to @code{EIO}.  Returns -1.

@item _read_r
Reads the specified number of bytes from the serial port using a
board-specific @code{inbyte} function.  Ignores the file descriptor
argument.  The @code{inbyte} function is not part of libgloss and must
be provided elsewhere.

@item _rename_r
Not implemented.

@item _sbrk_r
Grows the heap.  If the end of the heap extends beyond the
@code{_heap_sentry} symbol (set by the linker), sets @code{errno} to
@code{ENOMEM} and returns -1.  Note that unlike the ISS semihosting
version of @code{_sbrk_r}, this version does not currently check if the
heap collides with the stack.

@item _stat_r
Sets @code{errno} to @code{EIO} and returns -1.

@item _times_r
Returns the current cycle count value if the Xtensa processor
configuration includes the Timer Interrupt Option; otherwise, sets
@code{errno} to @code{ENOSYS} and returns -1.

@item _unlink_r
Sets @code{errno} to @code{EIO}.  Returns -1.

@item _wait_r
Not implemented.

@item _write_r
Writes the specified number of bytes to the serial port using
the board-specific @code{outbyte} function.  Ignores the file descriptor
argument.  The @code{outbyte} function is not included in libgloss and
must be provided elsewhere.
@end table

@node Xtensa ISS Syscalls
@subsection System calls with the Xtensa ISS

The Xtensa ISS semihosting library provides the following
system routines (all the routines using ISS semihosting set @code{errno}
and the return value based on the status of the host system call):

@table @code
@item _exit
Terminates the Xtensa ISS simulation.

@item _close_r
Closes the specified host file using the ISS semihosting interface.

@item environ
Not implemented.  It is assumed that this is provided in the
@samp{crt1-sim} file.

@item _execve_r
Not implemented.

@item _fcntl_r
Not implemented.

@item _fork_r
Not implemented.

@item _fstat_r
Assumes terminal I/O and sets @code{st_mode} to @code{S_IFCHR}.
Returns 0.

@item _getpid_r
Returns 1.

@item _gettimeofday_r
Gets the host system's current time expressed in elapsed seconds and
microseconds since January 1, 1970 using the ISS semihosting interface.

@item _kill_r
Calls @code{_exit} if the specified process ID is 1 (as returned by
@code{getpid}); otherwise, does nothing.  Returns 0.

@item _link_r
Adds a hard link to a host file using the ISS semihosting interface.
(Only supported on Unix platforms.)

@item _lseek_r
Seeks to the specified location in a host file using the ISS semihosting 
interface.

@item _open_r
Opens the specified file on the host using the ISS semihosting
interface.  The following flags are supported: @code{O_RDONLY},
@code{O_WRONLY}, @code{O_RDWR}, @code{O_APPEND}, @code{O_CREAT},
@code{O_TRUNC}, @code{O_EXCL}, @code{O_NDELAY}, @code{O_NONBLOCK},
@code{O_BINARY}, and @code{O_TEXT}.

@item _read_r
Reads from a host file using the ISS semihosting interface.

@item _rename_r
Renames a host file using the ISS semihosting interface.

@item _sbrk_r
Grows the heap.  If the end of the heap extends beyond the
@code{_heap_sentry} symbol (set by the linker) or if it collides with
the stack, sets @code{errno} to @code{ENOMEM} and returns -1.

@item _stat_r
Sets @code{errno} to @code{EIO} and returns -1.

@item _times_r
Returns the Xtensa simulation cycle count.

@item _unlink_r
Unlinks (i.e., removes) a host file using the ISS semihosting interface.

@item _wait_r
Not implemented.

@item _write_r
Writes to a host file using the ISS semihosting interface.
@end table

@node Xtensa gdbio
@subsection System calls with the gdbio library

The Xtensa ``gdbio'' library (libgdbio) implements the system routines as
described below.  Some of the routines are not implemented; programs using
libgdbio must avoid linking with functions that require those unimplemented
routines.

@table @code
@item _exit
Not implemented. It is assumed that this function is provided elsewhere,
typically in the @samp{crt1} file.

@item _close_r
Closes the specified host file using GDB's File-I/O protocol.

@item environ
Not implemented. The default implementation in newlib is used unless
another definition is provided elsewhere.

@item _execve_r
Not implemented.

@item _fcntl_r
Not implemented.

@item _fork_r
Not implemented.

@item _fstat_r
Gets the status of an open file using GDB's File-I/O protocol.

@item _getpid_r
Returns 1.

@item _gettimeofday_r
Gets the host system's current time expressed in elapsed seconds and
microseconds since January 1, 1970 using GDB's File-I/O protocol.

@item _kill_r
Calls _exit if the specified process ID is 1 (as returned by _getpid_r);
otherwise, does nothing. Returns 0.

@item _link_r
Not implemented.

@item _lseek_r
Seeks to the specified location in a host file using GDB's File-I/O protocol.

@item _open_r
Opens the specified file on the host using GDB's File-I/O protocol.  The
following flags are supported: @code{O_RDONLY}, @code{O_WRONLY},
@code{O_RDWR}, @code{O_APPEND}, @code{O_CREAT}, @code{O_TRUNC}, and
@code{O_EXCL}.

@item _read_r
Reads from a host file using GDB's File-I/O protocol.

@item _rename_r
Renames a host file using GDB's File-I/O protocol.

@item _sbrk_r
Grows the heap. If the end of the heap extends beyond the _heap_sentry
symbol (set by the linker), sets errno to ENOMEM and returns -1.
Note that unlike the ISS semihosting version of _sbrk_r, this version
does not currently check if the heap collides with the stack.

@item _stat_r
Gets the status of the specified host file using GDB's File-I/O protocol.

@item _times_r
Returns the current cycle count value if the Xtensa processor
configuration includes the Timer Interrupt Option; otherwise, sets
@code{errno} to @code{ENOSYS} and returns -1.

@item _unlink_r
Unlinks (i.e., removes) a host file using GDB's File-I/O protocol.

@item _wait_r
Not implemented.

@item _write_r
Writes to a host file using GDB's File-I/O protocol.

@item gdbio_system
This function provides access to the GDB File-I/O protocol's @code{system}
function.  It is renamed to avoid conflicting with newlib's standard
@code{system} function.  It is only supported on Unix platforms.
@example
int gdbio_system(const char *command);
@end example
The specified command is executed on the host by running
``@code{/bin/sh -c @var{command}}''.  The return value is -1 on error, or
the exit status of the command otherwise.  Note that this feature is
disabled by default in GDB and must be enabled by changing the
``remote system-call-allowed'' setting to a nonzero value.
@end table
@end ifset
